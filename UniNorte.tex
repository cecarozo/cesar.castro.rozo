\documentclass{article}\usepackage[]{graphicx}\usepackage[]{color}
%% maxwidth is the original width if it is less than linewidth
%% otherwise use linewidth (to make sure the graphics do not exceed the margin)
\makeatletter
\def\maxwidth{ %
  \ifdim\Gin@nat@width>\linewidth
    \linewidth
  \else
    \Gin@nat@width
  \fi
}
\makeatother

\definecolor{fgcolor}{rgb}{0.345, 0.345, 0.345}
\newcommand{\hlnum}[1]{\textcolor[rgb]{0.686,0.059,0.569}{#1}}%
\newcommand{\hlstr}[1]{\textcolor[rgb]{0.192,0.494,0.8}{#1}}%
\newcommand{\hlcom}[1]{\textcolor[rgb]{0.678,0.584,0.686}{\textit{#1}}}%
\newcommand{\hlopt}[1]{\textcolor[rgb]{0,0,0}{#1}}%
\newcommand{\hlstd}[1]{\textcolor[rgb]{0.345,0.345,0.345}{#1}}%
\newcommand{\hlkwa}[1]{\textcolor[rgb]{0.161,0.373,0.58}{\textbf{#1}}}%
\newcommand{\hlkwb}[1]{\textcolor[rgb]{0.69,0.353,0.396}{#1}}%
\newcommand{\hlkwc}[1]{\textcolor[rgb]{0.333,0.667,0.333}{#1}}%
\newcommand{\hlkwd}[1]{\textcolor[rgb]{0.737,0.353,0.396}{\textbf{#1}}}%
\let\hlipl\hlkwb

\usepackage{framed}
\makeatletter
\newenvironment{kframe}{%
 \def\at@end@of@kframe{}%
 \ifinner\ifhmode%
  \def\at@end@of@kframe{\end{minipage}}%
  \begin{minipage}{\columnwidth}%
 \fi\fi%
 \def\FrameCommand##1{\hskip\@totalleftmargin \hskip-\fboxsep
 \colorbox{shadecolor}{##1}\hskip-\fboxsep
     % There is no \\@totalrightmargin, so:
     \hskip-\linewidth \hskip-\@totalleftmargin \hskip\columnwidth}%
 \MakeFramed {\advance\hsize-\width
   \@totalleftmargin\z@ \linewidth\hsize
   \@setminipage}}%
 {\par\unskip\endMakeFramed%
 \at@end@of@kframe}
\makeatother

\definecolor{shadecolor}{rgb}{.97, .97, .97}
\definecolor{messagecolor}{rgb}{0, 0, 0}
\definecolor{warningcolor}{rgb}{1, 0, 1}
\definecolor{errorcolor}{rgb}{1, 0, 0}
\newenvironment{knitrout}{}{} % an empty environment to be redefined in TeX

\usepackage{alltt}

\usepackage[utf8]{inputenc}
\usepackage[english,spanish]{babel}
\usepackage[authoryear]{natbib}
\usepackage{fancyhdr}                % encabezados y pies
\usepackage{indentfirst}             % sangrías
\pagestyle{empty}
\usepackage{colortbl, xcolor}
\usepackage{hyperref}
\usepackage{eurosym}
\usepackage{amsmath}
\usepackage{tablefootnote}
\usepackage{float} 
\usepackage{authblk}
\usepackage{placeins}
\usepackage{amssymb}                 % additional math symbols
\usepackage{booktabs}
\usepackage{array}
\newcolumntype{L}[1]{>{\raggedright\arraybackslash}m{#1}}   % for table notes
\usepackage{appendix}
\usepackage{setspace} 
\setlength{\parskip}{1em}            % paragraph separation
\setlength{\parindent}{0pt}          % no indent
\IfFileExists{upquote.sty}{\usepackage{upquote}}{}
\begin{document}


UNIVERSIDAD DEL NORTE\\
Departamento de Economía
\vspace{10pt}

Proyección investigativa y docente\\
César Castro Rozo
\vspace{10pt}

23 de abril de 2018
\vspace{10pt}

Soy economista, con un marcado interés en temas de macroeconomía, econometría de series de tiempo y economía de la energía. Mi trayectoria profesional ha estado centrada principalmente en la investigación aplicada. Primero, trabajando durante cinco años como asistente de investigación y director en varios proyectos relacionados con macroeconomía y economía ambiental en el Centro de Investigaciones para el Desarrollo (CID) de la Universidad Nacional de Colombia (Bogotá). Posteriormente, trabajando por más de 12 años en el laboratorio de predicción y análisis macroeconómico del Instituto Flores de Lemus de la Universidad Carlos III de Madrid, con el equipo encargado de la publicación mensual de un boletín con análisis y predicciones de variables macroeconómicas de Estados Unidos, la euro área y sus principales economías. En este equipo participaba en el manejo de las bases de datos, el planteamiento y desarrollo de los modelos econométricos y la presentación de los resultados. La metodología se centraba en la implementación de modelos idiosincráticos -principalmente ARIMA- en estructuras jerárquicas desagregadas, incluye ndo indicadores adelantados, tratamientos estacionales, análisis de intervención, corrección de atípicos, etc.

Como parte de ésta investigación, obtuve mi PhD en la Universidad de Salamanca con la tesis ``Three Essays on Applied Time Series Econometrics''. Los tres capítulos que la componían se publicaron en revistas JCR: 2 en ``Energy Economics'' (Q1) y 1 en ``Economía Politica: Journal of Analytical and Institutional Economics'' (Q4). La tesis gira en torno al análisis de shocks exógenos, en concreto de los efectos de los shocks recientes en el precio del petróleo sobre variables de precios -industriales y al consumo- en la euro área y sus principales economías. Para ellos se utilizan modelos de funciones de transferencia, modelos VAR y técnicas de ``smoothing''. Posteriormente, con profesores de la Universidad Complutense de Madrid, he terminado una investigación sobre los efectos de estos shocks en el precio del petróleo sobre la inflación regional en España, utilizando funciones de transferencia (artículo actualmente bajo revisión en una revista JCR) y con mi directora de tesis en la Universidad de Salamanca, he terminado otro artículo sobre la relación entre tipos de cambio y precios de petróleo utilizando inferencia bayesiana (también bajo revisión en una publicación JCR). Actualmente me encuentro desarrollando dos investigaciones relacionadas con la política monetaria en la euro área y en Colombia.

En relación con la proyección investigativa y docente, considero que se deberían dirigir en dos líneas complementarias. En primer lugar, el manejo de datos se presenta como un reto para la investigación económica, en virtud de la mayor disponibilidad de información (calidad y cantidad en tiempo real), la mejora de los medios informáticos (``hardware'' y programas) y el avance en nuevas metodologías estadísticas (por ejemplo, inferencia bayesiana). En concreto, considero interesante el desarrollo de investigación en ``machine learning'', con la utilización de por ejemplo, lenguajes de programación como \textbf{\textsf{R}} (con el cual he trabajado los últimos 6 años) o \textbf{julia}. En segundo lugar, desde el punto de vista de la macroeconomía considero que después de la Gran Crisis se han abierto muchos interrogantes desde el punto de vista teórico y aplicado. A nivel internacional, las preguntas se centran en el papel de la política en el nuevo escenario macro, por ejemplo, la efectividad de la política monetaria frente a tasas de interés cero, el reforzamiento de la política fiscal ante la ausencia de auto-equilibrios de largo plazo o el papel de una nueva política financiera que haga frente a sus crecientes complejidades globales. Sin dejar de lado estas grandes discusiones, fuente de preocupación para las economías más desarrolladas, creo que se debería avanzar en paralelo sobre las consecuencias y retos que ha dejado la Gran Crisis para economías menos desarrolladas como la colombiana. Así, considero importante el estudio de temas como los efectos de los shocks externos (por ejemplo, precio del petróleo, política monetaria de la FED, etc.), los instrumentos y los efectos de la política macroeconómica en el nuevo contexto financiero mundial y las consecuencias que las nuevas políticas macroeconómicas globales (por ejemplo, el nuevo ordenamiento financiero) podrían tener en las economías menos desarrolladas. Como instrumentos para entender el nuevo escenario, resulta clave la integración de los hechos fácticos (por ejemplo, utilizando modelos VAR estructurales, Bayesianos estructurales de series de tiempo, etc.) con modelos DSGE.

\clearpage{}

UNIVERSIDAD DEL NORTE\\
Departamento de Economía\\
\vspace{10pt}

Propuesta de contenido para cursos de Macroeconomía\\
César Castro Rozo
\vspace{10pt}

\textbf{Macroeconomía Avanzada}
\begin{enumerate}
  \item Objetivo\\
  Estudiar técnicas econométricas recientes en series de tiempo aplicadas al manejo de datos macroeconómicos. Éstas técnicas abarcan todo el proceso de análisis de un problema macroeconómico, desde la obtención y tratamiento previo de los datos, hasta el análisis y pruebas finales necesarias para obtener resultados sólidos. El curso incluirá la instrucción y la práctica con el lenguaje de programación \textbf{\textsf{R}}.
  \item Contenido
  \begin{itemize}
    \item Modelos univariantes
    \item Simulación Montecarlo, bootstrapping, raíces unitarias
    \item Modelos multivariantes (VAR, VARMA, SVAR, cointegración, VECM) 
    \item Métodos de predicción
  \end{itemize}
  \item Metodología\\
  Clases magistrales, presentación de artículos por parte de los estudiantes y ejercicios aplicados con \textbf{\textsf{R}}.
  \item Bibliografía
  \begin{itemize}
    \item Enders, W. (2014). Applied Econometric Time Series. John, Wiley \& Sons.
    \item Hamilton, J.D. (1994), Time Series Analysis, Princeton, NJ: Princeton University Press.
    \item Kilian, L. Lütkepohl H. (2017). Structural Vector Autoregressive Analysis. Cambridge University Press.
    \item Lütkepohl, H. (2005, 2010), New Introduction to Multiple Time Series Analysis, 1st ed., New York: Springer-Verlag (Paperback).
    \item Shumway, R. H., Stoffer, D. S. (2016). Time Series Analysis and its Applications. Springer-Verlag
    \item Stock J, Watson MW. (2003). Introduction to Econometrics. New York: Prentice Hall. 
  \end{itemize}
\end{enumerate}

Algunos seminarios electivos adicionales que se podría desarrollar son:

\begin{itemize}
  \item \textbf{Economía del petróleo}\\
El objetivo es estudiar la dinámica de los mercados internacionales de materias primas, especialmente del petróleo y su importancia para la economía colombiana. Es un seminario de economía aplicada, utilizando técnicas macroeconométricas, en concreto análisis a través de modelos VAR. El contenido incluye el análisis de la estructura mundial del mercado, origen de los shocks, estudio de los efectos sobre las principales variables macroeconómicas (PIB e inflación), técnicas de predicción para los precios del petróleo y éstas variables macroeconómicas. La metodología consiste en clases magistrales, presentación de artículos por parte de los estudiantes y ejercicios aplicados con R.

  \item \textbf{Economía Europea}\\
El objetivo es estudiar el proceso histórico que ha llevado a la formación de la Comunidad Europea, la implementación de la moneda común, los problemas de coordinación de la política económica (en el contexto de la Gran Crisis) y la discusión actual de los retos de futuro. El contenido incluiría el estudio del mercado común, la unión monetaria y el análisis comparativo de la economía del bienestar. La metodología consiste en clases magistrales y presentación de artículos por parte de los estudiantes.
\end{itemize}

\end{document}
