\documentclass{article}\usepackage[]{graphicx}\usepackage[]{color}
%% maxwidth is the original width if it is less than linewidth
%% otherwise use linewidth (to make sure the graphics do not exceed the margin)
\makeatletter
\def\maxwidth{ %
  \ifdim\Gin@nat@width>\linewidth
    \linewidth
  \else
    \Gin@nat@width
  \fi
}
\makeatother

\definecolor{fgcolor}{rgb}{0.345, 0.345, 0.345}
\newcommand{\hlnum}[1]{\textcolor[rgb]{0.686,0.059,0.569}{#1}}%
\newcommand{\hlstr}[1]{\textcolor[rgb]{0.192,0.494,0.8}{#1}}%
\newcommand{\hlcom}[1]{\textcolor[rgb]{0.678,0.584,0.686}{\textit{#1}}}%
\newcommand{\hlopt}[1]{\textcolor[rgb]{0,0,0}{#1}}%
\newcommand{\hlstd}[1]{\textcolor[rgb]{0.345,0.345,0.345}{#1}}%
\newcommand{\hlkwa}[1]{\textcolor[rgb]{0.161,0.373,0.58}{\textbf{#1}}}%
\newcommand{\hlkwb}[1]{\textcolor[rgb]{0.69,0.353,0.396}{#1}}%
\newcommand{\hlkwc}[1]{\textcolor[rgb]{0.333,0.667,0.333}{#1}}%
\newcommand{\hlkwd}[1]{\textcolor[rgb]{0.737,0.353,0.396}{\textbf{#1}}}%
\let\hlipl\hlkwb

\usepackage{framed}
\makeatletter
\newenvironment{kframe}{%
 \def\at@end@of@kframe{}%
 \ifinner\ifhmode%
  \def\at@end@of@kframe{\end{minipage}}%
  \begin{minipage}{\columnwidth}%
 \fi\fi%
 \def\FrameCommand##1{\hskip\@totalleftmargin \hskip-\fboxsep
 \colorbox{shadecolor}{##1}\hskip-\fboxsep
     % There is no \\@totalrightmargin, so:
     \hskip-\linewidth \hskip-\@totalleftmargin \hskip\columnwidth}%
 \MakeFramed {\advance\hsize-\width
   \@totalleftmargin\z@ \linewidth\hsize
   \@setminipage}}%
 {\par\unskip\endMakeFramed%
 \at@end@of@kframe}
\makeatother

\definecolor{shadecolor}{rgb}{.97, .97, .97}
\definecolor{messagecolor}{rgb}{0, 0, 0}
\definecolor{warningcolor}{rgb}{1, 0, 1}
\definecolor{errorcolor}{rgb}{1, 0, 0}
\newenvironment{knitrout}{}{} % an empty environment to be redefined in TeX

\usepackage{alltt}

\usepackage[utf8]{inputenc}
\usepackage[english,spanish]{babel}
\usepackage[authoryear]{natbib}
\usepackage{fancyhdr}                % encabezados y pies
\usepackage{indentfirst}             % sangrías
\pagestyle{empty}
\usepackage{colortbl, xcolor}
\usepackage{hyperref}
\usepackage{eurosym}
\usepackage{amsmath}
\usepackage{tablefootnote}
\usepackage{float} 
\usepackage{authblk}
\usepackage{placeins}
\usepackage{amssymb}                 % additional math symbols
\usepackage{booktabs}
\usepackage{array}
\newcolumntype{L}[1]{>{\raggedright\arraybackslash}m{#1}}   % for table notes
\usepackage{appendix}
\usepackage{setspace} 
\setlength{\parskip}{1em}            % paragraph separation
\setlength{\parindent}{0pt}          % no indent
\IfFileExists{upquote.sty}{\usepackage{upquote}}{}
\begin{document}

\begin{center}
\textbf{UNIVERSIDAD PÚBLICA DE NAVARRA}\\
Facultad de Ciencias Económicas y Empresariales
\vspace{10pt}

Primer concurso ordinario de contratación de profesorado curso 2018/2019\\
\begin{tabular}{l l}
Plaza:  &Nº 4991\\
Categoría: &Profesor Ayudante Doctor\\
Departamento: &Economía\\
Área de Conocimiento: &Fundamentos del Análisis Económico\\
\end{tabular}

\textcolor[RGB]{85,87,89}{\rule{\linewidth}{0.4pt}}

\vspace{5pt}

\textbf{Proyecto docente}\\
César Castro Rozo\\
Mayo de 2018\\
\vspace{5pt}

\textcolor[RGB]{85,87,89}{\rule{\linewidth}{0.4pt}}

\begin{tabular}{l l}
Asignatura: &\textbf{MICROECONOMÍA I}\\
Tipo: &Básica\\
Curso: &I\\
Periodo: &Semestre 2\\
Departamento: &Economía
\end{tabular}
\end{center}
\vspace{5pt}

\begin{enumerate}
  \item Modulo/Materia\\
  Mención en Análisis Económico: Microeconomía

  \item Descriptores\\
        Teoría del productor, Teoría del consumidor, Costes, Preferencias, Oferta y Demanda, Equilibrio parcial

  \item Recomendaciones previas\\
        Introducción a la economía\\
        Economía de la empresa\\
        Conocimientos de excel

  \item Competencias genéricas\\
        CG01: Capacidad de análisis y síntesis\\
        CG03. Comunicación oral y escrita en la lengua nativa.\\
        CG05. Conocimientos de informática relativos al ámbito de estudio.\\
        CG07. Capacidad para la resolución de problemas.\\
        CG09. Capacidad para trabajar en equipo.\\
        CG17. Capacidad de aprendizaje autónomo.

  \item Competencias específicas\\
        CE04 Utilizar criterios profesionales para el análisis económico, preferiblemente aquellos basados en el manejo de instrumentos técnicos\\
        CE07: Contribuir a la buena gestión de la asignación de recursos tanto en el ámbito privado como en el público.\\
        CE09: Aportar racionalidad al análisis y a la descripción de cualquier aspecto de la realidad económica.

  \item Resultados aprendizaje\\
        Teoría de la producción: tecnologías y rendimientos a escala, costes, ofertas\\
        Teoría del consumidor: preferencias, elección, efecto renta y substitución, demandas\\
        Competencia: perfecta, imperfecta, monopolio, oligopolio

  \item Metodología
    \begin{itemize}
      \item Clases magistrales teóricas y prácticas en donde el profesor expone los temas claves y más complejos a través del uso de diapositivas que se pondrán a disposición del estudiante. Se buscará siempre la referencia a problemas reales de actualidad, sobre las cuales se plantearan ejercicios.
      \item Trabajos con ejercicios y pruebas teóricas no presenciales.
  \end{itemize}
  
  \begin{center}
  \begin{tabular}{ l  c  c  c }
  \hline
  & \multicolumn{3}{c}{Horas}\\
    \hline
      &Presenciales &No presenciales &TOTAL\\ \hline
    Clases teóricas &30 &30 &60\\
    Clases prácticas &24 &20 &44\\
    Preparación de trabajos & &16 &16\\
    Exámenes &6 &24 &30\\ \hline
    TOTAL &60 &90 &150\\ \hline
  \end{tabular}
\end{center}
      
      \item Relación actividades formativas-competencias
  \begin{center}
  \begin{tabular}{ l  l }
    \hline
    Actividad formativa &Competencias\\ \hline
    Clases teóricas &CG01-CG03-CG07-CG17-CE04-CE07-CE09\\
    Clases prácticas &CG01-CG05-CG07-CG09-CG17-CE04-CE07-CE09\\
    Preparación de trabajos &CG01-CG05-CG07-CG09-CG17-CE04-CE07-CE09\\
    Exámenes &CG01-CG05-CG07-CG09-CG17-CE04-CE07-CE09\\ \hline
  \end{tabular}
\end{center}
      
      \item Evaluación\\ 
      \\
  \begin{left}
  \begin{tabular}{ l  l }
    Pruebas parciales &20\%\\
    Trabajos &20\%\\
    Exámen final &60\%
  \end{tabular}
\end{left}

  \item Contenido
    \begin{itemize}
      \item Entender la racionalidad detrás de los problemas de decisión de productores y consumidores.
      \item Dentro de la teoría del productor, estudiar la función de producción, factores de producción y tecnología, diferencias de corto y largo plazo, tipos de costes y funciones de oferta.
      \item Dentro de la teoría del consumidor, estudiar la formación de preferencias, función de utilidad, relación marginal de sustitución, restricción presupuestaria y funciones de demanda.
\end{itemize}

  \item Temario
    \begin{enumerate}
      \item Teoría del productor
        \begin{itemize}
          \item Tecnologías y rendimientos de escala
          \item Curvas de costes y optimización
          \item Ofertas a corto y largo plazo
        \end{itemize}
      \item Teoría del consumidor
        \begin{itemize}
          \item Preferencias y utilidad
          \item Elección del consumidor y demanda
          \item Efectos renta y sustitución
          \item Elección intertemporal
          \item Demanda de mercado
        \end{itemize}
    \end{enumerate}
  
  \item Bibliografía
    \begin{itemize}
      \item Nicholson, W.: Microeconomía Intermedia y sus Aplicaciones. McGraw-Hill. 2001.
      \item Pindyck, R.S., Rubinfeld, D.L.: Microeconomía. Prentice Hall. 2003.
      \item Varian, H.R.: Microeconomía Intermedia: Un Enfoque Actual. Antoni Bosch. 2007
  \end{itemize}
\end{enumerate}

# ························································································································································

\vspace{5pt}
\begin{center}
\begin{tabular}{l l}
Asignatura: &\textbf{ECONOMETRÍA III}\\
Tipo: &Optativa\\
Departamento: &Economía
\end{tabular}
\end{center}

\begin{enumerate}
  \item Modulo/Materia\\
  Mención en Análisis Económico: Econometría.

  \item Descriptores\\
  Bases de datos, Modelos de series de tiempo, Modelos VAR, VECM, SVAR. Programación en \textbf{\textsf{R}}. Éstas técnicas son útiles para las áreas de especialización en análisis económico, desarrollo y cooperación y economía pública.

  \item Recomendaciones previas\\
        Econometría I, Econometría II\\
        Conocimientos de excel

  \item Competencias genéricas\\
        CG03. Comunicación oral y escrita en la lengua nativa.\\
        CG05. Conocimientos de informática relativos al ámbito de estudio.\\
        CG06. Habilidad para analizar y buscar información proveniente de fuentes diversas.\\
        CG07. Capacidad para la resolución de problemas.\\
        CG09. Capacidad para trabajar en equipo.\\
        CG16. Trabajar en entornos de presión.\\
        CG17. Capacidad de aprendizaje autónomo.

  \item Competencias específicas\\
        CMAE8: Capacidad para interpretar cualitativa y cuantitativamente los resultados de la estimación de modelos macroeconométricos y microeconométricos.

  \item Resultados aprendizaje\\
        Identificación, estimación de modelos económicos de series temporales.\\
        Análisis y predicción de variables macroeconómicas

      \item Metodología\\
        \begin{itemize}
          \item Clases magistrales
          \item Réplica de artículos por parte de los estudiantes
          \item Ejercicios aplicados
        \end{itemize}

   Las clases introducirán los aspectos más relevantes de cada tema. El estudiante preparará, replicará y presentará los resultados de un artículo relevante de alguno de los temas (similar a la metodología de la asignatura Economía Laboral) . Toda la práctica del curso se llevará a cabo con \textbf{\textsf{R}}. Uno de los objetivos del curso es lograr un manejo suficiente de \textbf{\textsf{R}} para el análisis y predicción de series temporales de variables macroeconómicas. Aunque existen programas estadísticos alternativos, \textbf{\textsf{R}} presenta ventajas relevantes, p.e. gratuidad, comunidad en continuo desarrollo, énfasis en el análisis estadístico de datos, multitud de paquetes especializados, programación intuitiva, investigación reproducible, etc. En especial, resulta interesante integrar dentro de una única plataforma (\textbf{\textsf{R}}/{\LaTeX}) el análisis estadístico de los datos y la elaboración de informes (incluídos cuadros y gráficos).

  \item Relación actividades formativas-competencias
  \begin{center}
  \begin{tabular}{ l  l }
    \hline
    Actividad formativa &Competencias\\ \hline
    Clases magistrales &CG03-CG07-CG17-CMAE8\\
    Réplica de artículos &CG05-CG06-CG07-CG09-CMAE8\\
    Ejercicios aplicados &CG05-CG06-CG07-CG17-CMAE8\\ \hline
  \end{tabular}
\end{center}
      
      \item Evaluación\\ 
      \\
  \begin{left}
  \begin{tabular}{ l  l }
    Réplica de artículos &70\%\\
    Ejercicios aplicados &30\%
  \end{tabular}
\end{left}

  \item Contenido\\
  Dentro de los objetivos del curso están:
    \begin{itemize}
      \item Estudiar técnicas econométricas en series de tiempo aplicadas al análisis y predicción de variables macroeconómicas.
      \item Proporcionar las herramientas necesarias para el estudio completo de un problema macroeconómico, incluídas la obtención y tratamiento previo de los datos, el análisis y pruebas finales necesarias para obtener resultados sólidos y la presentación de resultados (informes, cuadros, gráficos).
      \item Fortalecer la habilidad para sintetizar, analizar y llevar a cabo razonamientos lógicos deductivos e inductivos en torno a un problema de macroeconomía.
      \item Reforzar la capacidad para elegir la mejor técnica de análisis y predicción macroeconómica teniendo en cuenta criteros como el tipo de problema económico, frecuencia y disponibilidad de datos, la teoría económica, etc.
      \item Introducir el análisis estadístico utilizando el lenguaje de programación \textbf{\textsf{R}}.
    \end{itemize}

  \item Temario
    \begin{enumerate}
      \item Introducción al lenguaje de programación \textbf{\textsf{R}}
      \item Repaso modelos univariantes
        \begin{itemize}
          \item Procesos estocásticos, estacionariedad. ruido blanco, paseo aleatorio
          \item Funciones de Autocovarianzas y Autocorrelación total (ACF) y parcial (PACF), procesos lineales, MA, AR, ARMA, ARIMA
        \end{itemize}
      \item Modelos Dinámicos
        \begin{itemize}
          \item Modelos de retardos distribuidos, ARMAX
          \item Estimación de modelos dinámicos, Teorema de Mann-Wald
        \end{itemize}
      \item Contrastes de raíces unitarias y de estacionariedad (ADF, PP, KPSS)
      \item Simulación Montecarlo, bootstrapping
      \item Modelos multivariantes
        \begin{itemize}
          \item Modelos VAR, VECM, SVAR: Estimación y determinación del orden, funciones de impulso respuesta
          \item Relación entre modelos SVAR y modelos de ecuaciones simultáneas y DSGE
          \item Causalidad en el sentido de Granger
          \item Identificación de restricciones de corto y largo plazo
        \end{itemize}
  \end{enumerate}
  
  \item Bibliografía
  \begin{itemize}
    \item Enders, W. (2014). Applied Econometric Time Series. John, Wiley \& Sons.
    \item Hamilton, J.D. (1994), Time Series Analysis, Princeton, NJ: Princeton University Press.
    \item Kilian, L. Lütkepohl H. (2017). Structural Vector Autoregressive Analysis. Cambridge University Press.
    \item Lütkepohl, H. (2005, 2010), New Introduction to Multiple Time Series Analysis, 1st ed., New York: Springer-Verlag (Paperback).
    \item Shumway, R. H., Stoffer, D. S. (2016). Time Series Analysis and its Applications. Springer-Verlag
    \item Stock J, Watson MW. (2012). Introducción a la Econometría. 3a edición. New York: Prentice Hall. 
  \end{itemize}
\end{enumerate}

\end{document}
